\documentclass[11pt]{article}

\usepackage[a4paper, margin=1in]{geometry}
\usepackage{amsmath, amssymb, graphicx, natbib, hyperref}
\usepackage{authblk}
\usepackage{enumitem}

\title{Tempered Transitions and Annealing for Multimodal Posteriors}
\author[1]{Ezequiel de Braga Santos}
\affil[1]{School of Applied Mathematics - Getulio Vargas Foundation (FGV EMAp)}
\date{\today}

\begin{document}

\maketitle

\begin{abstract}

\end{abstract}

\section{Introduction}

\begin{itemize}
    \item Motivation: Why multimodal posteriors are challenging in Bayesian inference;
    \item Common problems: Local modes, poor mixing, label switching;
    \item Brief introduction to MCMC limitations in this context.
\end{itemize}



\section{Background}
\subsection{Multimodal Posteriors}
Explain why multimodality arises (e.g., in mixture models) and how it affects sampling.

\subsection{Limitations of Standard MCMC}
Discuss issues like poor mixing, high autocorrelation, and the label-switching problem.

\section{Tempered Transitions}
\subsection{Theory}
Tempered transitions \citep{neal1996sampling} involve defining a sequence of intermediate distributions between a flattened (high-temperature) and original (low-temperature) posterior.

\subsection{Algorithm}
Outline the algorithm steps or provide pseudocode.

\section{Simulated Annealing}
\subsection{Concept}
Simulated annealing introduces temperature control to help the chain escape local modes.

\subsection{Annealing Schedules}
Discuss temperature schedules and convergence guarantees.

\section{Applications}
\begin{itemize}
    \item \textbf{Mixture models}: Handling label switching.
    \item \textbf{Ancestral inference}: \citep{geyer1995annealing}.
    \item \textbf{Bayesian neural networks}.
\end{itemize}

\section{Implementation and Diagnostics}
Discuss:
\begin{itemize}
    \item Choosing a temperature ladder.
    \item Computational cost.
    \item Mixing diagnostics and visualization tools.
\end{itemize}

\section{Case Study: Gaussian Mixture Model}

\begin{itemize}
    \item Provide results comparing standard MCMC (Gibbs sample) vs. tempered transitions on a multimodal posterior;
    \item Analyse label-switching.
\end{itemize}


\section{Discussion}
Analyze strengths, weaknesses, and trade-offs. Suggest when these methods are useful and mention alternatives.

\section{Conclusion}
Summarize the benefits of tempered transitions and annealing for dealing with multimodal posteriors. Suggest future work or open problems.

\bibliographystyle{plainnat}
\bibliography{references}

\end{document}

